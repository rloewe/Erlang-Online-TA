\chapter{Correction Modules}
Correction modules using \texttt{gen\_assignment} is a vital part of the system
as they enable the addition of new sandboxes.

In this project we created two different modules, one using docker and one using
Safe Haskell. The reason we chose those two was to show that these two very
different types of "sandboxes" were possible to implement.

The first module we made was the docker module. In the \texttt{setup} function,
it makes a Dockerfile which installs all the required packages, creates a user
called correction and copies the correction scripts. It then returns a command
that the \texttt{gen\_assignment} module will run.
The \texttt{teardown} function simply removes the docker image that was created.
In the \texttt{run} function, it is check whether disk and/or network is allowed
before it gives back a command, which starts a container with the correct
properties and the submission directory mapped to it.
\lstinputlisting{../src/lib/docker.erl}

The Safe Haskell module is a little smaller. In both the \texttt{setup} and
\texttt{teardown} functions both gives back \texttt{done} as nothing is required
to set up or remove it. In the \texttt{run} function, it creates a command which
copies the correction scripts into the handin folder compiles it with ghc and
runs what comes out.
\lstinputlisting{../src/lib/safehaskell.erl}
