\chapter{Analysis}
The main challenge here is to create a distributed server structure that can
handle load, which most likely changes over time. To handle this a master-slave
architecture could be a good choice. Here the master server should handle
everything related to receiving and distributing assignments and submissions and
the slave should correct them. To make the infrastructure more dynamic, slaves
servers should be able to connect and disconnect at will, so that increasing or
decreasing workload can be easily handled.

\section{Master server}
As said before, the master server should handle everything related to receiving
and distributing assignments and submissions. This means it has to make sure that we
create a somewhat equal load on all our nodes and that we monitor our nodes, so
we do not send submissions to non-responding servers. It also has to make sure
that submissions send to a node, which crashes before the submission is corrected,
is send to another node, such that the submission does not disappear in the
system.
Hand ins might also need to be handled with a fairness system in mind, such that
students can not create a DoS attack by uploading their submissions several
times in a row.
The master should also be able to handle different types of assignments, which
means that there should be a way that these can be described and some way of
authorizing the users who should be able to upload such.
The assignments will then have to be distributed to all nodes, also new nodes
which join at a later stage.

\section{Node server}
The node server has a main goal, which is to correct submissions, but it also
needs to handle receiving assignments and queuing submissions. It also needs to
make sure that the assignments are run in a secure manner, such that a student
will no be able to read other students submissions and/or mess with the system.
This can be fixed in multiple ways and might be different for each assignment,
therefore the system should be able to handle this.
Error handling is also something the nodes should do, for example a submission
should be able to run again, if a server error happens, but should be stopped if
it fails multiple times. This could for example be fixed by monitoring processes
and restarting them, if they fail.

\section{Sandboxing}
The system should be able to handle several types of sandboxes, as one type of sandbox
might not be suitable for all types of assignments or required in some cases. It should
be easy to add additional sandboxes to the system, with the possibility to restrict access to
specific resources in the sandbox, e.g. disabling disk or network access. It should also be
possible to limit them in other ways, for example by setting a max runtime or max memory usage.
