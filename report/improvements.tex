\section{Improvements}
The code we have made, in this project, is not complete nor perfect. Therefore
we have this part of our report to discuss new features or improvements to the
project.

\subsection{Remove/update assignments and modules}
Currently it is not possible to update or remove assignments and modules. This
should be a feature in the future.

\subsection{Better error handling}
The error handling in the code is currently very poor. There are many places in
the code where we only match the good scenarios and do not match if it gives an
error back.

Places where it is especially urgent to address is when a module is added and
when an assignment is build. Currently niether of these two places in the code
checks for errors and assumes that everything is okay. To fix the error handling
when a module is added, we should probably take an erl file in stead of a beam
file, which we currently do, and compile it using \texttt{code:compile}.
Fixing the error handling when an assignment is build, should be a lot easier
as it should just use a \texttt{case} on the call and have a \texttt{try-catch}
around it to catch the errors.

\subsection{DoS protection/fairness algorithm}
As written in the analysis, DoS protection through a scheduling algorithm with
fairness would have been a great thing to implement in this project.

\subsection{Better interface}
Currently the web-interface for the project is very sparse and not very good.
It basically just shows that we are able to do some communication with the
master server. This should definitely be improved such that different people get
access to different parts of the system, e.g. lecturers should be able to upload
assignments and students should be able to submit assignments.

Another thing would be to design an interface that is nice to look at. But as we
are Computer Science students, we normally think command line interfaces are
good enough and therefore we should probably not be the ones to design it.

\subsection{Date range for assignments}
A nice feature to add would be to add a date range for the assignments, such
that they only are active during that period. Of course it should also be
possible to circumvent that date range for certain people, so they can test it
before it launches.

\subsection{Possibility to get correction output from file}
Currently it is only possible to get correction output from \texttt{stdout}. It
would be nice if it was also possible to get that output from a file. This
should be implemented as an extra field in the assignment configuration and then
the correction scripts that are being run should then write that file.

\subsection{Monitor \texttt{correct\_fsm}s and \texttt{gen\_assignment}s}
The \texttt{correct\_fsm} and the \texttt{gen\_assignment} are vital parts of
the system, though currently they are not monitored. So if they crash, they are
not restarted. This can lead to problems, because the master just assumes that
every node is running in perfect condition, if it as not gone down.

\subsection{Ability to run scripts outside of the sandbox}
As some scripts in the correction process can be set to be run outside of the
sandbox with the keyword \texttt{unsafe}. Currently it is up to the module to
decide whether it will do that or not, but it would be nice if it was not up to
the correction module to handle this as it will make it easier to write such
modules.

\begin{itemize}
    \item security, auth (restrict assignments to specific users)
    \item supervision trees
\end{itemize}
